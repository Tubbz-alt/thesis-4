\Section{intro}{Introduction}
We approach the task of common-sense reasoning by casting it
  as a \textit{database completion} task: given a non-exhaustive
  database of true facts, we would like to predict whether an unseen fact
  is true and should belong in the database.
This is intuitively cast as an inference problem from 
  a collection of candidate premises to the truth of the query.
For example, we would like to infer 
  that \textit{no carnivores eat animals}
  is false given a database containing \textit{the cat ate a mouse}
  (see \reffig{teaser}).

\begin{figure}[th]
\begin{center}
  \teaserSearch
\end{center}
\caption{
  Natural Logic inference cast as search. The path to
    the boxed premise \w{the cat ate a mouse} disproves the
    query \w{no carnivores eat animals}, as it passes through
    the negation relation ($\negate$).
  This path is one of many candidates taken; the premise
    is one of many known facts in the database.
  The edge labels denote Natural Logic inference steps.
  \label{fig:teaser}
}
\end{figure}

These inferences are difficult to capture in a principled way
  while maintaining high recall, particularly for 
  large scale open-domain tasks.
Learned inference rules are difficult to generalize to arbitrary
  relations, and standard IR methods easily miss small but
  semantically important lexical differences.
Furthermore, many methods require explicitly modeling either the
  database, the query, or both in a formal meaning representation
  (e.g., Freebase tuples).

Although projects like the Abstract Meaning Representation
  \cite{key:2013banarescu-amr} have made headway in providing
  broad-coverage meaning representations, it remains 
  appealing to use human language as the vessel for
  inference.
Furthermore, OpenIE and similar projects have been very successful at
  collecting databases of natural language snippets
  from an ever-increasing corpus of unstructured text.
These factors motivate our use of Natural Logic -- a proof system built
  on the syntax of human language -- for
  broad coverage database completion.

Prior work on Natural Logic has focused on inferences from a single
  relevant premise, making use of only
  formally valid inferences.
We improve upon computational Natural Logic in three ways:
  (i) our approach operates over a very large set of
    candidate premises simultaneously;
  (ii) we do not require explicit alignment between a premise 
    and the query;
  and (iii) we allow imprecise inferences at an
    associated cost learned from data.
%    to provide a broad-coverage system over real-world facts.

%In a similar vein, prior work has made use of alignments between the
%  premise and the conclusion, which is not present in our setting;
%  we show that we can do well on
%  valid Natural Logic inferences even without this explicit alignment.
%Lastly, we allow imprecise inferences%, with an associated confidence,
%  to provide a more broad-coverage system over real-world facts.

Our approach casts inference as a single unified search problem from 
  a query to any valid supporting premise.
Each transition along the search denotes a (reverse) inference step 
  in Natural Logic, and incurs a cost reflecting the system's confidence
  in the validity of that step.
This approach offers two contributions over prior work in
  database completion:
  (i) it allows for unstructured text as the input database without
    any assumptions about the schema or domain of the text,
  and (ii) it proposes Natural Logic for inference, rather than
    translating to a formal logic syntax.
Moreover, the entire pipeline is implemented 
  in a single elegant search framework, which scales easily to large
  databases.
%  within the context
%  of a single search over these inferences.

%
%   ---
%   VERSION 2 SNIPPETS
%   ---
%

%However, prior work on Natural Logic has a few shortcomings:
%  (i) it assumes we know the premise for our inference,
%  (ii) it requires explicit alignment between this premise and the
%       query,
%  and (iii) it enforces strict entailment, whereas many applications
%    could make use of probabilistic entailment accompanied by a
%    confidence.
%We address the first two shortcomings by formulating Natural Logic
%  inference as a search problem from a query to any supporting
%  premise.
%The third point is addressed by allowing imprecise inference steps,
%  at the cost of an associated learned penalty.
% 
%
%Natural Logic allows us to reason using human language without
%  appealing an intermediate logical form.
%Although projects like the Abstract Meaning Representation
%  \cite{key:2013banarescu-amr} have made headway providing a
%  broad-coverage logical representation for language, it remains 
%  appealing to use the human language itself as the vessel for
%  inference.
%This is particularly true for common sense types of facts, where
%  the meaning is often less crisp than in Freebase-style
%  factoids.
%%In particular, Natural Logic is well-suited to capture many types
%%  of facts which are difficult to formalize 
%
%A key application for natural language inference is, in turn,
%  database completion: given a set of facts, predict whether an
%  unseen fact should belong in the database.
%Many such databases -- e.g., output from the OpenIE
%  project \cite{key:2007banko-openie} -- are in reality semi-structured
%  or unstructured text.
%This makes Natural Logic an appealing tool for the task.
%
%However, prior work on Natural Logic has a few shortcomings:
%  (i) it assumes we know the premise for our inference,
%  (ii) it requires explicit alignment between this premise and the
%       query,
%  and (iii) it enforces strict entailment, whereas many applications
%    could make use of probabilistic entailment accompanied by a
%    confidence.
%We address the first two shortcomings by formulating Natural Logic
%  inference as a search problem from a query to any supporting
%  premise.
%The third point is addressed by allowing imprecise inference steps,
%  at the cost of an associated learned penalty.
%
%This then allows us to efficiently infer whether an unseen
%  common sense fact is \textit{true} or \textit{false} on the basis
%  of a very large, noisy, unstructured collection of known facts.
%An example inference is shown in \reffig{teaser};
%  \refsec{maccartney} explains the inference steps in the search,
%  \refsecs{search}{inference} describe the search problem in detail.

%
%   ---
%   VERSION 1
%   ---
%


%Common-sense reasoning if prevalent in a number of AI tasks;
%  although such reasoning is difficult in general, we hope to
%  capture a subset of useful phenomena.
%For instance, in computer vision it may be useful to provide priors
%  for cars generally being found on roads, or
%  for people not often drinking grass.\needcite
%Similarly, for robotics (e.g., for illustration, a robot to help in
%  the kitchen) it may be useful to know that milk is found
%  in a refrigerator, or that tomatoes are soft and 
%  easy to crush.\needcite
%
%
%As a step in this direction, we created NaturaLI: a reasoning engine
%  based around Natural Logic to infer the truth of arbitrary query
%  facts given a large database of known facts.
%The system is intended to follow valid logical derivations when possible,
%  but also back off gracefully to dubious inferences with an
%  associated confidence of validity.

%Prior work has largely focused on freebase-style factoids:
%  for instance, \textit{Barack Obama is married to Michelle}, or
%  \textit{Natasha is the daughter of Michelle Obama}.
%In this framework, inference tasks are generally cast as expanding
%  a partial knowledge base with additional high-probability facts
%  (e.g., Knowledge Base Population).
%While this approach is viable for these sorts of factoid entries,
%  the number of common-sense facts -- the vast majority of which are
%  never explicitly mentioned -- is unlikely to be amenable to this
%  strategy.
%
%We therefore re-cast the problem as querying whether an arbitrary fact
%  is true or not, given the entire knowledge base as the
%  antecedent set.
%We then approach the problem as finding a correct reverse derivation
%  from the consequent to any antecedent in our knowledge base.
%
%When possible, these derivations should be valid Natural Logic derivations;
%  however, the search is robust to proposing plausible, if not
%  strictly correct, inferences at a discount proportional to the
%  likelihood of validity.
%For instance, our approach generalizes similarity metrics, suggesting
%  that if two entities are similar to each other (e.g., cats and dogs)
%  then they are likely to share properties (e.g., have tails).
%This discount on incorrect inferences can be set to enforce strict
%  validity, but can as easily be trained on a dataset of facts labeled
%  with their truth, to calibrate the resulting probability returned
%  from the system.
%
%We evaluate the system on the FraCaS entailment suite to motivate its
%  validity as an inference engine, and evaluate in a more realistic
%  setting of predicting the truth of ReVerb extractions
%  \cite{key:2011fader-reverb} annotated with ground truth correctness.
