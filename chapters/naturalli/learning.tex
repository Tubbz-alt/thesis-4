\Section{learning}{Learning Transition Costs}
%The learning task can be viewed as a constrained optimization problem.
We describe our procedure for learning the transition costs $\theta$.
Our training data $\sD$ consists of query facts $q$ and their associated
  gold truth values $y$.
Equation \refeqn{prob} gives us a probability that a particular inference
  is \textit{valid}; we axiomatically consider a valid inference from
  a known premise to be justification for the truth of the query.
This is at the expense of the (often incorrect) assumption that our 
  database is clean and only contains true facts.
%  note that this is subtly different from our objective of
%  determining if the query is \textit{true}.
%Nonetheless, we take these two ($p(\textit{valid})$ and $p(\textit{true})$) to be
%  equivalent for our experiments, which corresponds to the (often
%  incorrect) assumption that our database is clean and only contains
%  true facts.
%trivially, we define the probability that a
%  fact is true as the probability that an inference from a premise to
%  that fact is valid.
%  we assume that every fact in our database is true, which allows us
%  to use the equation as a proxy for the probability that a fact
%  is true.

We optimize the likelihood of our gold annotations according to
  this probability, subject to
  the constraint that all elements in our cost vector $\theta$ be
  non-negative.
We run the search algorithm described in \refsec{inference} on every
  query $q_i \in \sD$.
This produces the highest confidence path $x_1$, along with its
  inference state $v_i$.
We now have annotated tuples: $( (x_i, v_i), y_i )$ for every
  element in our training set.
Analogous to logistic regression, the log likelihood of our training
  data $\sD$, subject to costs $\theta$, is:

%Subject to the constraint that all elements of the cost vector $\theta$ 
%  must be non-negative, we optimize the probability from
%  Equation \refeqn{prob} compared against the gold annotation.
%As training data, we are given a number of facts, annotated with a
%  truth value of \textit{true} or \textit{false}.
%We assume that all facts in our database are true; therefore,
%  $p(\textrm{valid})$ corresponds directly to $p(\textrm{true})$.
%
%We learn costs using an iterative algorithm.
%At each iteration, we take the costs from the previous iteration
%  and run search over every example to obtain candidate paths.
%We then construct a supervised training set $\sD$ with entries
%  $((x_i, v_i), y_i)$ corresponding to the predicted path $x_i$,
%  the predicted inference state $v_i$, and the ground truth $y_i$.
%Analogous to logistic regression,
%  the log-likelihood of the training data becomes:

\vspace{0.25cm}
\begin{align*}
\l_\theta(\sD&) = \sum_{0 \leq i < |\sD|} \Big[
    y_i \log \left(\frac{v_i}{2} + \frac{1}{1 + e^{v_i \theta \cdot \bef(x_i)}} \right) \\
    &+ (1 - y_i) \log \left(\frac{-v_i}{2} + \frac{1}{1 + e^{-v_i \theta \cdot \bef(x_i)}} \right)
  \Big],
\end{align*}
\vspace{0.25cm}

\noindent where $y_i$ is 1 if the example is annotated true and 0
  otherwise, and $\bef(x_i)$ are the features extracted for path $x_i$.
The objective function is the negative log likelihood with
  an $L_2$ regularization term and a log barrier function to 
  prohibit negative costs:

\vspace{0.25cm}
\begin{equation*}
O(\sD) = -\l_\theta(\sD) 
  + \frac{1}{2 \sigma^2}\|\theta\|_2^2
  - \epsilon \log(\theta).
\end{equation*}
\vspace{0.25cm}

We optimize this objective using conjugate gradient descent.
Although the objective is non-convex, in practice we can find a good
  initialization of weights to reduce the risk of arriving at
  local optima.

An elegant property of this formulation is that the weights we are
  optimizing correspond directly to the costs used during search.
This creates a positive feedback loop -- as better weights are learned,
  the search algorithm is more likely to find confident paths, and
  more data is available to train from.
We therefore run this learning step for multiple epochs,
  re-running search after each epoch.
The weights for the first epoch are initialized to an approximation
  of valid Natural Logic weights.
Subsequent epochs initialize their weights to the output of the previous
  epoch.


%Therefore, we opt to \naive ly run a convex solver on the objective after
%  each iteration, initialized to the weights from the previous
%  iteration.
%The weights for the first iteration are initialized to an approximation
%  of valid Natural Logic weights.

%We initialize the first iteration to these costs, and run
%  conjugate gradient descent to convergence on a local minima
%  on each subsequent iteration.
%Lastly, after each iteration, the cost for any feature which has
%  never fired during training is decreased by half.
%This is to encourage the search to take these expensive edges, and
%  learn an appropriate cost for them.



%We learn costs using an iterative algorithm.
%At each iteration, we take the costs from the previous iteration
%  and run the derivation search over every example, providing
%  a predicted probability of truth for each query fact.
%Optimizing the likelihood of the training data according to
%  Equation \refeqn{prob} is impractical as the objective is nonconvex;
%  the rest of today is intended to be spent on implementing a piecewise
%  optimization to mitigate this.
%A simple heuristic (weight = 1 - \# times this feature fired in good examples
%  / \# of times feature fired total) does ok, but is a little braindead.
%
%In particular, we construct a dataset of $(x,y)$ pairs, where $x$ is the
%  featurized path for a particular example, and $y$ is whether the
%  path ended in a correct state.
%A correct state is defined trivially between \textit{valid} and
%  \textit{invalid}.
%In cases where the gold annotation contains a state of 
%  \textit{unknown validity}, any prediction is marked incorrect.
%
%The resulting weights -- in one-to-one correspondence with the costs
%  for our search -- are then negated and normalized via a bilinear
%  transform to fall between 0 and $-5$, with a mean of $-1$.
%A default weight of $-2$ is assigned for any feature which did not
%  appear in the training data, excepting the case where a previous
%  iteration of the algorithm had set a value for that weight, in
%  which case the previous value is used.
