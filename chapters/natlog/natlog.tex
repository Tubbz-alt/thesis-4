%
% MONOTONICITY
%
\Section{natlog-mono}{Monotonicity Calculus}


Monotonicity in natural logic may be viewed analogously to monotonicity
  in algebra.
More precisely, a function can be one of \textit{monotone},
  \textit{antitone}, or \textit{non-monotone} in each of its
  arguments.\footnote{
    \textit{Monotone} and \textit{antitone} are often referenced
    as \textit{upwards monotone} and \textit{downwards monotone}.
  }
Intuitively, the function $f(x,y,z) = x - y + (z-2)^2$ is
  monotone in $x$, antitone in $y$, and non-monotone in $z$.
At its heart, natural logic makes use of this monotonicity information
  and the partial ordering over the domain and range of the function
  to draw inferences over the formula for $f$ without evaluating
  any part of the function explicitly.
For instance, we can safely say that $f(1,1,1) \leq f(2,1,1)$, whereas
  $f(1,1,1) \geq f(1,2,1)$ by appealing directly to surface form of
  the expression (The substitution of `$2$' in place of `$1$').

Formally, we define a function $f:\sD_1 \rightarrow \sD_2$ and a partial
  ordering $\le_1$ over $\sD_1$ and $\le_2$ over $\sD_2$.
If we set $\sD_1 = \sD_2 = \sR$, and take $\le_1 = \le_2 = \le$ to be
  the conventional ``less than or equal to'' over real numbers,
  we can derive the monotonicity of the functions in our example above.
%For instance, $f(x) = x + c$ is monotone; $g(x) = -x$ is antitone;
%  $h(x) = (x-2)^2$ is non-monotone.

However, we can equally well define a partial ordering over sets:
  $x \le_e y \Leftrightarrow x \cap y = x$.
Similarly, we can define a partial ordering over functions
  $f : \sD_1 \rightarrow \sD_2$:
  $f \le_1 g \Leftrightarrow \forall x ~~ f(x) \leq_2 g(x)$
  and truth values:
  $x \le_t y \Leftrightarrow x = F \lor y = T$.
This allows us to adapt our definition of monotonicity to denotational
  semantics in natural language.

Let the denotation of a phrase, marked \denote{\cdot}, be the set of
  entities in the universe to which that phrase refers.
Equivalently, we may consider each phrase as a predicate, and the denotation
  of a phrase is the set of entities for which the predicate holds.
We define the set $e$ to be the set of entities in our discourse space,
  with the partial ordering over sets: $\le_e$.
We define the set $t$ to be the set of truth values: $\{T, F\}$.
It should not go unnoticed that Modus Ponens ($x \Rightarrow y$)
  and $x \le_t y$ are equivalent.
Note that functions $e \rightarrow t$ follow the partial ordering over
  functions.

We define the denotation of quantifiers in the conventional
  fashion as functions $e \rightarrow (e \rightarrow t)$.
However, we can further mark the arguments to these quantifiers as
  potentially monotone or antitone.
For instance, the quantifier \textit{all} is antitone in its first
  argument and monotone in its second:
  $e \xrightarrow{-} (e \xrightarrow{+} t)$.\footnote{
    These markings can, in fact, be elegantly incorporated into the
    type system \cite{key:2014icard-natlog} of a CCG.
  }
These markings confirm our intuition that, e.g., 
  \textit{all \textbf{cats} have tails}
  implies 
  \textit{all \textbf{tabby cats} have tails}.
Furthermore, as the default context is taken to be monotone
  axiomatically, we see that we can appeal to the partial order
  over quantifiers (functions) to conclude that
  \textit{\textbf{all} cats have tails}
  implies 
  \textit{\textbf{some} cats have tails}.

\paragraph{Proof Theory}
We can produce proofs in natural logic by composing
  function application from atomic facts.
Returning to the arithmetic cast in the beginning of the section, we
  can construct a proof that $2^{-4} < 2^{-3}$ via:

\begin{prooftree}
\AxiomC{$3 \leq 4$}
\RightLabel{\scriptsize{$-x$ is antitone}}
\UnaryInfC{$-4 \leq -3$}
\RightLabel{\scriptsize{$2^x$ is monotone}}
\UnaryInfC{$2^{-4} \leq 2^{-3}$}
\end{prooftree}

We can then appeal implicitly to this proof to ask whether what relation
  a substitution of $3$ for $4$ will produce in $2^{-4}$ -- in this case
  one of $\leq$, $\geq$, or unknown -- without evaluating either
  expression.


Returning to natural language, we can provide a proof for
  \textit{all tabby cats have tails}, given in \reffig{catsproof}.
That is to say, on the given parse of the respective sentences,
  \denoteT{All cats have tails} $\leq_t$ \denoteT{All tabby cats have tails}.
We again note that $\leq_t$ is precisely Modus
  Ponens ($\Rightarrow$) -- thus, we have arrived at our expected
  entailment relation.

\begin{figure*}
  \begin{prooftree}
  \AxiomC{\denoteT{tabby cats} $~\le_e~$ \denoteT{cats}}
  \RightLabel{\scriptsize{\denoteT{All} antitone in subject}}
  \UnaryInfC{$
    \denoteT{all}\left(\denoteT{cats}\right)\left(\cdot\right)
    ~\le_{e \rightarrow t}~
    \denoteT{all}\left(\denoteT{tabby cats}\right)\left(\cdot\right)
  $}
  \AxiomC{\denoteT{have tails} $~\le_e~$ \denoteT{have tails}}
%  \RightLabel{\scriptsize{\denoteT{All} monotone in object}}
  \BinaryInfC{$
    \denoteT{all}\left(\denoteT{cats}\right)\left(\denoteT{have tails}\right)
    ~~\le_{t}~~
    \denoteT{all}\left(\denoteT{tabby cats}\right)\left(\denoteT{have tails}\right)
  $}
  \RightLabel{\scriptsize{By axiom}}
  \UnaryInfC{$
    \denoteT{all cats have tails}
    ~~\le_{t}~~
    \denoteT{all tabby cats have tails}
  $}
  \end{prooftree}
\longcaption{A proof in natural logic for \textit{all tabby cats have tails} from
    the premise \textit{all cats have tails}.}
{\label{fig:catsproof}
  A proof in natural logic for \textit{all tabby cats have tails} from
    the premise \textit{all cats have tails}.
  Note that $\le_t$ is equivalent to entailment ($\Rightarrow$).
  The last step, marked valid axiomatically, is elaborated on in
    Definition 8 of \newcite{key:2014icard-natlog}.
} 
\end{figure*}


%
% EXCLUSION
%
\Section{natlog-exclusion}{Reasoning with Exclusion}
The natural logic of \refsec{natlog-mono} has dealt only with the
  entailment relation.
Work by \newcite{key:2009maccartney-natlog}, building off of
  \newcite{key:2007maccartney-natlog} and
  \newcite{key:2008maccartney-natlog},
  has focused on extending the set of atomic relations
  beyond $\leq$ (subset; entailment) and 
  $\geq$ (superset; reverse entailment).
%We adopt the semantics of \newcite{key:2012icard-natlog}, although in this
%  work for simplicity we do not make use of their extended
%  monotonicity markings.
%A more thorough treatment of the soundness and completeness of the
%  proof theory described here can be found in
%  \newcite{key:2013djalali-natlog}.

\newcite{key:2014icard-natlog} noted that the partial ordering of
  $\leq_e$ can be formulated as a \textit{bounded distributive
  lattice} to encompass the original relations of
  \newcite{key:2007maccartney-natlog}.
In particular, for every domain we can define a set of possible 
  elements $\sX$, binary operators $\land$ and $\lor$,
  and a minimum and maximum element $\bot$ and $\top$ respectively.
By example, for the domain of entities we can define $\sX$ to be
  the power set of our domain of entities, $\land=\cap$,
  $\lor=\cup$, $\bot=\{\}$, and $\top=E$, where $E$ is the universe
  of possible entities.
We then define the set of MacCartney relations, noting that
  $\forward$ and $\reverse$ are identical to the definitions of
  $\lte_e$ and $\gte_e$ from the previous section respectively:\footnote{
    We adopt the semantics of \newcite{key:2012icard-natlog}, 
    in which the relations are not mutually exclusive.
  }

\begin{center}
\begin{tabular}{lcl}
  $x \forward y$ & $\Leftrightarrow$ & $x \land y = x$ \\
  $x \reverse y$ & $\Leftrightarrow$ & $x \lor y = x$ \\
  $x \alternate y$ & $\Leftrightarrow$ & $x \land y = \bot$ \\
  $x \cover y$ & $\Leftrightarrow$ & $x \lor y = \top$ \\
  \vspace{-0.5em} & & \\
  $x \equivalent y$ & $\Leftrightarrow$ & $x \forward y~~\textrm{and}~~x \reverse y$ \\
  $x \negate y$ & $\Leftrightarrow$ & $x \alternate y~~\textrm{and}~~x \cover y$ \\
  $x \independent y$ & & Always true
\end{tabular}
\end{center}

Making use of this extended set of relations, we are faced with two
  questions:
  we must define how monotonicity projects these relations up the
  proof tree,
  and we must define how a sequence of edits compose to determine the
  relation between the first and last sentence in the sequence.

% -- Projection Table --
% I'm assuming the context is additive and multiplicative
% See Lemma 2.4 (p.11) of http://link.springer.com/article/10.1007%2Fs11225-012-9425-8
\begin{table}
	\begin{center}
	\begin{tabular}{c|cc}
    \textbf{Relation} & \multicolumn{2}{|c}{\textbf{Projection in Context}} \\
             & Monotone & Antitone \\
    \hline
    $e_1 \forward     e_2$ & $t_1 \forward     t_2$ & $t_1 \reverse     t_2$ \\ 
    $e_1 \reverse     e_2$ & $t_1 \reverse     t_2$ & $t_1 \forward     t_2$ \\ 
    $e_1 \alternate   e_2$ & $t_1 \alternate   t_2$ & $t_1 \cover t_2$ \\ 
    $e_1 \cover       e_2$ & $t_1 \cover       t_2$ & $t_1 \alternate       t_2$ \\ 
    $e_1 \equivalent  e_2$ & $t_1 \equivalent  t_2$ & $t_1 \equivalent  t_2$ \\ 
    $e_1 \negate      e_2$ & $t_1 \negate      t_2$ & $t_1 \negate   t_2$ \\ 
    $e_1 \independent e_2$ & $t_1 \independent t_2$ & $t_1 \independent t_2$
	\end{tabular}
	%(caption)
	\longcaption{The natural logic projection table.}
  {
    The projection table used for a relation between a phrase $e_1$ and
      its candidate mutation $e_2$, in terms of the produced relation
      between truth value $t_1$ and the new truth value $t_2$.
    Values are given for when the entities are in a monotone and
      antitone context.
		\label{tab:projectivity}
	}
	\end{center}
\end{table}

\reftab{projectivity} outlines the (simplified) table used for projecting
  relations between entities in monotone and antitone contexts.
Note that in monotone contexts, the relation projects up without
  change;
  in antitone contexts, forward and reverse entailment
  ($\forward$ and $\reverse$) are flipped, as are cover and
  alternation ($\cover$ and $\alternate$).
In full generality, quantifiers are tagged not only with monotonicity
  but also \textit{additivity} and \textit{multiplicativity};
  for simplicity we assume all quantifiers are both additive and
  multiplicative.
See Lemma 2.4\ in \newcite{key:2012icard-natlog} for a more careful
  analysis.

% -- Join Table--
\begin{table}
	\begin{center}
	\begin{tabular}{|c||c|c|c|c|c|}
    \hline
    $\bowtie$ & $\forward$ & $\reverse$ & $\negate$ & $\alternate$ & $\cover$ \\
    \hline
    $\forward$ & $\forward$ & $\independent$ & $\alternate$ & $\alternate$ & $\independent$ \\
    $\reverse$ & $\independent$ & $\reverse$ & $\cover$ & $\independent$ & $\cover$ \\
    $\negate$ & $\cover$ & $\alternate$ & $\equivalent$ & $\reverse$ & $\forward$ \\
    $\alternate$ & $\independent$ & $\alternate$ & $\forward$ & $\independent$ & $\forward$ \\
    $\cover$ & $\cover$ & $\independent$ & $\reverse$ & $\reverse$ & $\independent$ \\
    \hline
	\end{tabular}
	%(caption)
	\longcaption{The natural logic join table.}
  {
    The join table, as copied from \newcite{key:2012icard-natlog}.
    Note that the $\independent$ always joins to yield $\independent$,
    and $\equivalent$ always joins to yield the input relation.
		\label{tab:join}
	}
	\end{center}
\end{table}

The composition of relations through an inference chain is given in
  a \textit{join table}, copied in \reftab{join}.

%Note that unlike in the original MacCartney formulation, these relations
%  are not mutually exclusive; in fact, a hierarchy of informativeness
%  emerges where, e.g., \negate subsumes \alternate and \cover.
%
%We are now faced with two issues which need resolving to generalize
%  the proofs from \refsec{natlog-mono}.
%First, we need to resolve how the monotonicity of the function application
%  projects the relation in the argument.
%Implicitly used in the proofs so far is the notion that antitone contexts
%  will flip $\leq$ and $\geq$, whereas monotone contexts will not.
%This relation is preserved for $\forward$ and $\reverse$.
%\newcite{key:2012icard-natlog} present an enriched tagging of 
%  monotonicity which allows the other relations (i.e.,
%  \negate, \alternate, \cover) to project meaningfully.
%This work does not make use of this semantics, although it is not
%  theoretically impossible to do so -- practically, this prohibits
%  our system from reasoning about antonyms.
%
%The second issue is determining how these relations project through
%  the proof tree.


\begin{figure*}
\begin{center}
  \begin{tabular}{cc}
    \resizebox{0.48\textwidth}{!}{\completeFSA} &
      \resizebox{0.48\textwidth}{!}{\collapsedFSA} \\
    (a) & (b)
  \end{tabular}
\end{center}
\longcaption{The natural logic join table expressed as a finite state automata.}
{
  \label{fig:fsa}
  (a) The join table in \reftab{join} expressed as a finite state automata.
  Omitted edges go to the unknown state (\independent), with the exception of
    omitted edges from $\equivalent$, which go to the state of the edge
    type.
  Green states (\equivalent, \forward) denote valid inferences;
    red states (\alternate, \negate) denote invalid inferences;
    blue states (\reverse, \cover) denote inferences of unknown validity.
  (b) The join table collapsed into the three meaningful states over truth
  values.
}
\end{figure*}

A novel contribution of this dissertation is the observation that
  the join table can be cast as a finite state automata (see \reffig{fsa}a).
The edges in the automata denote a step in a logical derivation;
  for instance, attempting to derive \textit{all cats have tails}
  from \textit{all felines have tails}.
The label of the edge denotes the relation between the mutated
  \textbf{entity} in the derivation.
In our example, \textit{felines} $\forward_e$ \textit{cats},
  and therefore we would take the edge labeled with \forward.

States in the automata denote the relation between the
  \textbf{truth value} of
  the antecedent, and the truth value of the consequent given the
  mutations so far.
We may consider as a motivating example the inference from
  \textit{all felines have tails} to \textit{no cats have tails}:

\vspace{1em}
\noindent\resizebox{0.48\textwidth}{!}{
  \begin{tabular}{llcc}
  \textbf{Mutation} & \textbf{Fact} & \textbf{Edge} & \textbf{State} \\
  & \textit{all felines have tails} & & \equivalent \\
  \textit{feline} \reverse\ \textit{cat} & \textit{all cats have tails} & \forward & \forward \\
  \textit{all} \negate\ \textit{no} & \textit{no cats have tails} & \negate & \alternate
  \end{tabular}
}
\vspace{1em}

The second column shows the surface form of the fact as the derivation
  progresses.
The first column shows the mutation applied to the previous fact to get
  the new surface form.
The third column shows the edge to take in the FSA -- this is derived
  from the projections in \reftab{projectivity}.
For example, since \textit{all} is antitone in its first argument
  (\textit{felines}), a mutation of type \reverse\ projects to an edge
  type of \forward.
The first three columns can be viewed as shorthand for a
  formal proof tree like that provided in \reffig{catsproof}.
The last column shows the relationship between the first fact
  and the fact on this row of the derivation.
In our example derivation, we conclude that:

\begin{center}
  \textit{all felines have tails}
  \alternate\ \textit{no cats have tails}.
\end{center}

%
% COLLAPSED STATES
%
\Section{natlog-collapsed}{Collapsed Join Table}
Although the full join table is useful when we would like fine-grained
  semantic comparisons, for our application we care only whether
  an inference is \textit{valid}, \textit{invalid}, and \textit{unknown}.
To illustrate, we do not care that \textit{all felines have tails}
  is in any particular set theoretic relation with
  \textit{no cats have tails}; we only care that this is a verifiably
  invalid inference.

To this end, we make two novel contributions to the theory of Natural
  Logic:
first, we show that the join table can be simplified in the case where
  we only care about the validity of an inference; and, second, we
  propose weighting the edges in this finite state automata in such
  a way as to collapse the \textit{unknown} state into one of 
  either \textit{valid} or \textit{invalid}, with low probability.

The first of these contributions, illustrated in \reffig{fsa}b,
  makes use of two observations.
First, if we cluster the seven MacCartney relations\footnote{
  The seventh relation (\independent) is not shown in the FSA for
  clarity, but clusters into the unknown class.
  }, the transitions into and out of each class is identical independent
  of which specific state one is in.
This clustering -- trivial intuitively -- is also the clustering used
  by \newcite{key:2008maccartney-natlog} for evaluating inference
  tasks.

Furthermore, although there are transitions between the states clustered
  into the unknown category, at no point is there a valid transition
  from a node in the cluster to a node outside of it.
This is a generalization of the well-known observation that long
  derivations tend to the unknown state (\independent), from which
  it is impossible to escape.
Here, we observe that even if a derivation enters a seemingly-informative
  state (\reverse, \cover) it is nonetheless doomed to produce a
  final judgment of unknown.

The second observation -- that we can run probabilistic inference and
  collapse the unknown state into a penalty on the probability of
  validity/invalidity -- is discussed in the next section.

%
% LIMITATIONS
%
\Section{natlog-limitations}{Limitations and Future Work}
natural logic is not intended to be a complete solution for natural
  language inference.
Although the formalism covers a reasonable range of intuitive inference
  patterns, the current theory is unable to capture the full range
  of logical inferences.
For example, DeMorgan's laws 
%  ($\lnot(A \lor B) \models \lnot A \land \lnot B$),
  cannot be inferred in the framework,
  including their extensions to quantification
  (e.g., \textit{not all students study} $\models$ \textit{some students don't study}).
Furthermore, natural logic does not elegantly handle inferences
  making use of multiple antecedents; for instance, the law of
  excluded middle is not derivable according to current theory.

\newcite{key:2008maccartney-natlog} point out that natural logic is
  not amenable to reasoning over relational implication
  (\textit{Eve was let go} $\models$ \textit{Eve lost her job};
   \textit{Aho, a trader at UBS} $\models$ \textit{Aho works for UBS};
   etc.).
We contest that these are special cases of monotone reasoning, where
  relations are treated as functions and ordered in the same way as
  quantifiers.
That is, if a relation $r_1$ entails another relation $r_2$ then
  $r_1 \leq_r r_2$, where $\leq_r$ is an ordering of functions from
  a list of entities to truth values: $\be^n \rightarrow \{T,F\}$.
In fact, under certain conditions natural logic can warrant inference
  over meronymy
  (\textit{Obama was born in Hawaii} $\models$ \textit{Obama was born in the US}),
  and potentially other orderings as well.
We leave this for future work.


