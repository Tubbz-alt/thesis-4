\Section{natlog-propositional}{A Propositional Natural Logic}

So far all of our natural logic proofs have had a single premise.
This is not accidental -- a key shortcoming of monotonicity calculus and modern natural
  logics is that by their very nature they operate on a single premise.
That premise is mutated in well-specified ways which preserve entailment; but there's no
  theory of how to combine multiple premises together for an inference.
For example, the disjunctive syllogism is not supported by natural logic:


\[
\begin{nd}
\hypo {1} {\ww{Either Bill or Ted stole the cookies.}}
\hypo {2} {\ww{Ted did not steal the cookies.}}
\have {3} {\ww{Bill stole the cookies.}}
\end{nd}
\]

This section sketches a theoretical approach for performing these inferences by taking a hybrid
  propositional and natural logic.
In this way, we can leverage the relative strengths of both formalisms.
Propositional logic has rich notions of the common connectives: conjunction, disjunction, etc.;
  monotonicity calculus is complementary to propositional logic in its handling of quantifiers
  and lexical entailment.
At a high level, we would like a propositional logic to handle inferences between natural language
  propositions, while deferring to a natural logic to handle the entailments within propositions.

We briefly review propositional logic, and then show how we can perform hybrid natural logic
  and propositional proofs.
%Lastly, we explore how we might build a practical system to parse a sentence into propositional
%  components, and the expected challenges therein.


%
%
%
\Subsection{natlog-propositional-review}{Propositional Logic}
In propositional logic, we have propositions (usually denoted by capital letters), and connectives
  between these propositions.
A proposition has a truth value in a given model; propositions are things like ``cats have tails''
  or ``the sky is blue''.
A special proposition denotes contradiction ($\bot$) -- an axiomatically false statement.
The connectives in propositional logic are a subset of the connectives in first order logic:
  conjunction ($\land$), disjunction ($\lor$), and negation ($\lnot$).
Propositional logic does not have predicates (e.g., $P(x)$), and we will not be considering
  equality and identity (i.e., $P = Q$ or $x = y$).
Furthermore, for simplicity, we will treat material implication axiomatically in terms of
  the core connectives:

\begin{center}
  $ \lnot A \lor B \equivalent A \rightarrow B$
\end{center}

This section is not intended to be a formal introduction to propositional proofs,
  but rather a review of the \textit{natural deduction} system and Fitch-style
  proofs we will use in the remainder of this section 
  \cite{key:1934jaskowski-deduction,key:1935gentzen-deduction1,key:1935gentzen-deduction2}.
In particular, for each of our connectives (and the contradiction symbol), we define a rule
  for when we can \textit{introduce} that connective, and when we can \textit{eliminate} the
  connective.
In the next section, we will show how we can augment these rules with monotonicity calculus.


%
%
%
\paragraph{Conjunction Rules ($\land$)}
The two rules for conjunction are quite straightforward.
We can always take as true either side of a conjunction, and we can introduce a conjunction
  only if we can prove both conjuncts:

\begin{center}
\begin{tabular}{c@{\hspace{2cm}}c}
  $
  \begin{nd}
  \hypo {1} {A \land B}
  \have {2} {A}            \by{$\land$ E}{1}
  \end{nd}
  $
&
  $
  \begin{nd}
  \hypo {1} {A}
  \hypo {2} {B}
  \have {3} {A \land B}   \by{$\land$ I}{1,2}
  \end{nd}
  $ \\

  $
  \begin{nd}
  \hypo {1} {A \land B}
  \have {2} {B}            \by{$\land$ E}{1}
  \end{nd}
  $
& \\
\\
\textbf{$\land$ elimination} & \textbf{$\land$ introduction}
\end{tabular}
\end{center}


%
%
%
\paragraph{Disjunction Rules ($\lor$)}
Conjunction is, of course, quite boring on its own.
Disjunction is a measure more interesting.
Introducing disjunction is straightforward: if we know $A$, we also know that either $A$ or $B$.
However, eliminating a disjunction requires reasoning by cases.
We have to show that the same formula is true if it is true for all the disjuncts.
Formally, the rules are:

\begin{center}
\begin{tabular}{c@{\hspace{2cm}}c}
  $
  \begin{nd}
  \hypo {1} {A \lor B}
  \open[]
    \hypo         {2} {A}
    \have[\vdots] {3} {\vdots}
    \have[n]      {4} {Q}
  \close
  \open[]
    \hypo         {5} {B}
    \have[\vdots] {6} {\vdots}
    \have[m]      {7} {Q}
  \close
  \have           {8} {Q}        \by{$\lor$ E}{4,7}
  \end{nd}
  $
&
  $
  \begin{nd}
  \hypo {1} {A}
  \have {3} {A \lor B}   \by{$\lor$ I}{1}
  \end{nd}
  $ \\
\\
\textbf{$\lor$ elimination} & \textbf{$\lor$ introduction}
\end{tabular}
\end{center}


%
%
%
\paragraph{Negation Rules ($\lnot$)}
Negation elimination is straightforward: we can eliminate double negations.
Negation can be introduced, in turn, via a proof by contradiction:

\begin{center}
\begin{tabular}{c@{\hspace{2cm}}c}
  $
  \begin{nd}
  \hypo {1} {\lnot \lnot A}
  \have {2} {A}        \by{$\lnot$ E}{1}
  \end{nd}
  $
&
  $
  \begin{nd}
  \open[]
    \hypo         {2} {A}
    \have[\vdots] {3} {\vdots}
    \have[n]      {4} {\bot}
  \close
  \have           {5} {\lnot A}        \by{$\lnot$ I}{2-4}
  \end{nd}
  $ \\
\\
\textbf{$\lnot$ elimination} & \textbf{$\lnot$ introduction}
\end{tabular}
\end{center}


%
%
%
\paragraph{Contradiction Rules ($\bot$)}
Contradiction can be introduced when we derive a proposition and its negation.
Contradiction elimination is clearly the least intuitive of the deduction rules:
  if we have derived a contradiction, we can state any proposition as true.
The Fitch rules for contradiction are:

\begin{center}
\begin{tabular}{c@{\hspace{2cm}}c}
  $
  \begin{nd}
  \hypo {1} {\bot}
  \have {2} {Q}        \by{$\bot$ E}{1}
  \end{nd}
  $
&
  $
  \begin{nd}
  \hypo         {2} {A}
  \have[\vdots] {3} {\vdots}
  \have[n]      {4} {\lnot A}
  \have         {5} {\bot}        \by{$\bot$ I}{2,4}
  \end{nd}
  $ \\
\\
\textbf{$\lnot$ elimination} & \textbf{$\lnot$ introduction}
\end{tabular}
\end{center}

The next section shows how we can adapt these rules to the incorporate natural logic.



%
%
%
\Subsection{natlog-propositional-hybrid}{A Hybrid Logic}

In the previous section, we treated propositions as atomic units.
A formula $A \lor B$ has no substructure beyond being a disjunction of two
  propositions.
However, in most cases, these propositions \textit{do} in fact have structure.
Furthermore, most of the time we can express a proposition as a sentence.
In this section we propose an extension to propositional logic which uses monotonicity
  calculus to augment the semantics of these atomic propositions (conversely: an extension
  to monotonicity calculus which incorporates propositional reasoning).

Recall that in natural logic, a sentence is represented as a truth value.
Whereas \denote{cat} is a set of entities which are cats, and verbs like \denote{run}
  are denoted as functions from sets to truth values, a complete sentence
  (e.g., \denote{cats run}) is either \textit{true} or \textit{false}.
We can treat any denotation in natural logic which falls into this domain of
  \{true, false\} as propositions for propositional reasoning.
From this point, the propositional logic side of our proof theory no longer has to care about
  the substructure of the sentence, and can treat it as an atomic proposition.

Notationally, we define the space of well formed formulas in our hybrid logic analogously to how a
  formula is defined in propositional logic:

\begin{enumerate}
\item \denote{x} is a well-formed formula iff $\denote{x} \in \{true, false\}$.
      \denote{cat} is not a well-formed formula, but \denote{cats have tails} is.
\item $\bot$ is a well-formed formula.
\item If $F$ is a well-formed formula, so is $\lnot F$.
\item If $F$ and $G$ are well-formed formulas, so is $F \land G$.
\item If $F$ and $G$ are well-formed formulas, so is $F \lor G$.
\end{enumerate}


Having defined our logical language, we now turn to our hybrid proof theory.
Trivially, all of the inference rules of propositional logic hold, if we do not mutate
  any of the atomic sentences.
Likewise, trivially all of the natural logic inferences hold, if they are not embedded in
  a larger propositional formula.
Our task is then twofold:
  (1) what natural logic inferences are warranted inside of a given propositional formula; and
  (2) what \textit{additional} propositional inferences can be made by appealing to the semantics
      of natural logic.

The first of these questions is answered by noticing that the propositional connectives,
  like natural language quantifiers, have monotonicity.
Conjunction ($\land$) is upward monotone, and both additive and
  multiplicative; disjunction ($\lor$) is upward monotone, but 
  only additive.
Conjunction is multiplicative because 
  $(A \land B) \land C \vDash (A \land C) \land (B \land C)$,
  and additive because
  $(A \lor B) \land C \vDash (A \land C) \lor (B \land C)$.
Analogously, disjunction is not multiplicative:
  $(A \land B) \lor C \nvDash (A \land C) \lor (B \land C)$;
  but it is additive:
  $(A \lor B) \lor C \vDash (A \lor C) \lor (B \lor C)$.
Negation, in turn, is downward monotone, and anti-multiplicative but not anti-additive:
  $\lnot (A \lor B) \vDash \lnot A \lor \lnot B$, but
  $\lnot (A \land B) \nvDash \lnot A \land \lnot B$.

From here, we can use our normal projection rules to mutate lexical items not only in a sentence,
  but also embedded in a propositional formula.
For example, the following is now a valid inference using only natural logic:

~\newline
$
\begin{nd}
\hypo {1} {\lnot \denote{all cats are friendly} \land \denote{all cats are cute}}
\have {2} {\lnot \denote{all \textbf{felines} are friendly} \land \denote{all cats are cute}}        \by{\denote{cat} \forward\ \denote{feline}}{1}
\have {3} {\lnot \denote{all felines are friendly} \land \denote{all \textbf{tabby cats} are cute}}        \by{\denote{cat} \reverse\ \denote{tabby cat}}{2}
\end{nd}
$
~\newline

Of course, we should be able to now apply the propositional rules 
  from \refsec{natlog-propositional-review}:

~\newline
$
\begin{ndresume}
\have {4} {\denote{all tabby cats are cute}}        \by{$\land$ E}{3}
\end{ndresume}
$
~\newline


Perhaps the most interesting question is, what \textit{additional} inferences can we 
  draw from this hybrid logic?
In particular, our two logics each have their own notion of negation:
  in monotonicity calculus, \negate, \alternate, and \cover\ each behave a bit like negation.
In fact, this lets us define a new negation introduction rule in propositional logic.
In short, if we derive a new sentence from a presupposed true premise, and it is in the natural logic
  relations \negate\ or \alternate\ with the original sentence, we can introduce the negation
  of the original sentence.
Analogously, if we derive a new sentence from a presupposed false premise, and it is in the natural logic
  relations \negate\ or \cover\ with the original sentence, we can introduce the negation
  of the original sentence (to yield a double negation).
This falls directly out of the state formulation of the finite state automata
  described in \reffig{natlog-fsa}.
Formally:



\begin{center}
\begin{tabular}{c@{\hspace{2cm}}c}
  $
  \begin{nd}
  \hypo         {1} {\denote{X}}         \by{state:$\Rightarrow$}{}
  \have[\vdots] {2} {\vdots}
  \have[n]      {3} {\denote{Y}}         \by{state:$\lnot \Rightarrow$}{1-3}
  \have         {4} {\lnot \denote{X}}   \by{$\lnot$ I}{1,3}
  \end{nd}
  $
&
  $
  \begin{nd}
  \hypo         {1} {\lnot \denote{X}}         \by{state:$\Rightarrow$}{}
  \open[]
  \hypo         {2} {\denote{X}}         \by{state:$\lnot \Rightarrow$}{1}
  \have[\vdots] {3} {\vdots}
  \have[n]      {4} {\denote{Y}}         \by{state:$\Rightarrow$}{2-4}
  \close
  \have         {5} {\lnot \lnot \denote{X}}   \by{$\lnot$ I}{1,2-4}
  \end{nd}
  $ \\
\\
\textbf{natural $\lnot$ introduction 1} & \textbf{natural $\lnot$ introduction 2}
\end{tabular}
\end{center}


The added complication here is that we now have to keep track of whether introduced
  formulas are presupposed to be true or false.
For instance, line 2 in the right $\lnot$ introduction proof presupposes that the sentence
  \denote{X} is false, and therefore the cover (\cover) relation can flip the truth of the sentence
  to be true.
Similarly, in the left proof we assume that our premises are true -- this is not necessarily the case
  if, e.g., you are trying to prove your premises false.\footnote{
    This is a problem even in vanilla natural logic proofs. If you believe the premise to be
    false, you have to start in the false state of the finite state automata in
    \reffig{natlog-fsa}.
  }
If this information is not available \textit{a priori} in the proof, the new negation introduction
  rules are only warranted by the full negation relation (\negate).

\begin{figure}[th]
\begin{center}
\[
\begin{nd}
\hypo {1} {\denote{Bill stole the cookies} \lor \denote{Ted stole the cookies}}
\hypo {2} {\denote{Ted did not steal the cookies}}
\open[]
  \hypo {3} {\denote{Bill stole the cookies}}
  \have {4} {\denote{Bill stole the cookies}}        \r{3}
\close
\open[]
  \hypo {4} {\denote{Ted stole the cookies}}
  \have {5} {\denote{Ted did steal the cookies}}     \by{\denote{steal} \equivalent\ \denote{did steal}; state:$\Rightarrow$}{4}
  \have {6} {\denote{Ted did not steal the cookies}} \by{\denote{did} \negate\ \denote{did not}; state:$\lnot \Rightarrow$}{5}
  \have {7} {\lnot \denote{Ted stole the cookies}}   \by{$\lnot$ I}{4,6}
  \have {8} {\bot}                                   \by{$\bot$ I}{4,7}
  \have {9} {\denote{Bill stole the cookies}}        \by{$\bot$ E}{8}
\close
\have {10} {\denote{Bill stole the cookies}}         \by{$\lor$ E}{1,3-4,4-9}
\end{nd}
\]
\longcaption{A hybrid propositional + natural logic proof showing the disjunctive syllogism.}
{\label{fig:natlog-propositional-simpleproof}
  A hybrid propositional + natural logic proof showing the disjunctive syllogism.
}
\end{center}
\end{figure}

We can combine these insights to solve a our motivating example of the disjunctive
  syllogism, shown in \reffig{natlog-propositional-simpleproof}.



%
%
%
\Subsection{natlog-propositional-syntax}{Shallow Semantic Parsing}
The hybrid proof system from \refsec{natlog-propositional-hybrid} lays out an elegant theory
  for entailment when we are composing atomic sentences.
However, in natural language, propositional statements are usually not said in a such a straightforward
  way.
Even taking our simple running example, it's much more natural to say
  \ww{Either Bill or Ted stole the cookies} than \ww{Bill stole the cookies or Ted stole the cookies}.
Therefore, it is clearly useful to have a semantic parser which takes as input a complex
  utterance, and produces as output a hybrid natural/propositional logic formula.
Such a parser would allow us to do powerful forms of logical reasoning, without requiring
  a heavyweight logical representation (e.g., the Abstract Meaning Representation 
  \cite{key:2013banarescu-amr}).
In this section, we outline some of the challenges for such a parser.


\paragraph{\ww{And} is not always conjunction}
The classic case of this is a sentence like \ww{Jack and Jill are friends}, where of course
  this sentence should not be parsed as \ww{Jack is friends} and \ww{Jill is friends}.
However, there are more subtle cases of this as well.
For example, consider the exchange:

\begin{displayquote}
\ww{What would you like for dinner?} \\
\ww{Taco Bell and Pizza Hut are my choice.}
\end{displayquote}

Here, the utterance \ww{Taco Bell and Pizza Hut are my choice} has a meaning more akin to
  \ww{Taco Bell is my choice} or \ww{Pizza Hut is my choice}.


\paragraph{\ww{Or} is not always disjunction}
Conversely, \ww{or} does not always imply a disjunction.
For example, consider the exchange:

\begin{displayquote}
\ww{Can I have pets in the apartment?} \\
\ww{Cats or dogs are ok.}
\end{displayquote}

Clearly here, the semantics of the second statement is: 
  \denote{cats are ok} $\land$ \denote{dogs are ok}.


\paragraph{\ww{Or} is often exclusive}
In propositional logic, disjunction is not exclusive: $A \lor B$ does not
  forbid $A \land B$.
This is played out in many uses of the word \ww{or}; for example:
  \ww{you should do your taxes or your homework}.
However, in a surprising number of cases, \ww{or} implies an exclusive or.
Take, for instance:

\begin{displayquote}
\ww{Do or do not; there is no try.} \\
\ww{Either cats or dogs chase mice.} \\
\ww{Jack or Jill is responsible for that.}
\end{displayquote}

Handling these and other phenomena in language is a nontrivial problem.
However, this is true for any semantic parsing task; and if a good parser can be built
  then propositional and natural logic provide a fast and expressive paradigm for solving
  entailment tasks.


