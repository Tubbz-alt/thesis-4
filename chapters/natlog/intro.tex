A key underlying formalism throughout much of this thesis is
  the theoretical framework laid by natural logics.
Broadly, natural logics aims to capture a subset of valid logical
  inferences by appealing directly to the structure of language,
  as opposed to running deduction on an abstract logical form.
That is to say, a natural logic is primarily a \textit{proof theory}, where
  the deductive system operates over natural language syntax.

%
% Example
%
To illustrate with an example, we can consider a simple premise:

\begin{center}
\ww{The cat ate a mouse.}
\end{center}

From this, we would like to logically infer the following:

\[
\begin{nd}
\hypo {1} {\ww{The cat ate a mouse.}}
\have {2} {\ww{The cat ate a rodent.}}
\have {3} {\ww{The cat ate an animal.}}
\have {4} {\ww{The feline ate an animal.}}
\have {5} {\ww{The carnivore ate an animal.}}
\have {5} {\lnot ~~ \ww{No carnivore ate an animal.}}
\end{nd}
\]

That is to say, if the cat ate a mouse, it is false that no carnivore ate an
  animal, and this is by virtue of the proof presented above.
This sort of proof appears strange from the perspective of propositional and
  first-order logics, but it is not an altogether unfamiliar sort of reasoning.
Natural logics trace their origins to the syllogistic reasoning of Aristotle:

\[
\begin{nd}
\hypo {1} {\ww{All cats eat mice.}}
\hypo {2} {\ww{All cats are carnivores.}}
\have {3} {\ww{All carnivores eat mice.}}
\end{nd}
\]

%
% Outline
%
This section will describe some of the motivations for natural logic,

%
% Syllogisms
%
We \cite{key:1986benthem-natlog,key:1991valencia-natlog}.

%
% Sanchez Valencia-style Logic
%

A working approximation for natural logic is as a principled formalism
  for and extension of syllogistic reasoning.
In this analogy, the minor premise
  (e.g., \textit{All Greeks are men}) is encoded implicitly,
  and an inference is made from the major premise to the consequent:
  (e.g., \textit{All men are mortal} $\Rightarrow$ \textit{All Greeks are mortal}).
This style of reasoning is warranted from an analysis of the monotonicity
  of quantifiers, reviewed in \refsec{natlog-mono}.

The additional expressive power of natural logic is derived from two
  sources: in part, the logic does not rely on template matching,
  allowing inferences to be made over patterns nested within a larger
  context, and allowing inferences to be chained.
In another part, we can make use of MacCartney's expanded entailment
  relations, described in \refsec{natlog-exclusion}.
We conclude the section with a novel result showing that the state
  space of the MacCartney-style derivations can be collapsed to a
  simpler representation (\refsec{natlog-collapsed}), and by
  elaborating on the limitations of natural logic (\refsec{natlog-limitations}).

%We briefly motivate this analogy in \refsec{natlog-syllogism},
%  review recent theory on natural logic in
%  \refsecs{natlog-mono}{natlog-exclusion}, and propose a collapsed
%  represenation of MacCartney's entailment relations in
%  \refsec{natlog-collapsed}.
%Syllogistic reasoning can be considered a first approximation to
%  natural logic, as is presented in \refsec{natlog-syllogism}.
%A brief introduction beyond this first approximation is presented
%  in the subsequent sections.
%\refsec{natlog-mono} introduces \textit{monotonicity} in the context
%  of natural language to form the basis of natural logic;
%  \refsec{natlog-exclusion} extends this formalism for reasoning about
%  a wider range of phenomena.
\newcite{key:2014icard-natlog} offers a more thorough summary of the
  field, and provides the source for much of the notation and
  exposition presented here.
