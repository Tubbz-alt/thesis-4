\prefacesection{Preface}
This thesis describes a new technique for learning open-domain knowledge from unstructured web-scale text corpora,
  making use of a probabilistic relaxation of natural logic -- 
  a logic which uses the syntax of natural language as its logical formalism.
We begin by reviewing the theory behind natural logic, and propose a novel extension of the logic to handle
  propositional formulae.

We then show how to capture common sense facts: given a candidate statement about the world and a large corpus of 
  known facts, is the statement likely to be true? 
This is treated as a search problem 
  from the query statement to its appropriate support in the knowledge base over valid (or approximately valid) 
  natural logical inference steps.
This approach achieves a 4x improvement at retrieval recall compared to lemmatized lookup, 
  maintaining above 90\% precision.

We then extend the approach to handle longer, more complex premises by segmenting these utterance into a set of 
  atomic statements entailed through natural logic.
We evaluate this system in isolation by using it as the main component in an Open Information Extraction system, 
  and show that it achieves a 3\% absolute improvement in F1 compared to prior work on a competitive knowledge 
  base population task.

Finally, we address how to elegantly handle situations where we could not find a supporting premise for our query.
To address this, we create an analogue of an evaluation function in gameplaying search: a shallow lexical 
  classifier is folded into the search program to serve as a heuristic function to assess how likely we would 
  have been to find a premise.
Results on answering 4th grade science questions show that this method improves over both the classifier in isolation, 
  a strong IR baseline, and prior work.
