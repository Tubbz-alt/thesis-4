In this chapter, we review the body of work most related to this dissertation, and to
  the larger goal of extracting knowledge from text.
We cover related work in extracting structures knowledge bases from text
  (Knowledge Base Population) and open domain knowledge base population
  as two methods for extracting structured and semi-structured knowledge from
  text.
We then review work on common-sense reasoning -- one of the main goals of this dissertation.
Lastly, we review work on textual entailment, related to the underlying natural logic
  proof system in this dissertation, and conclude with reviewing work on question answering.


%
% KBP + RE
%
\Section{related-kbp}{Knowledge Base Population}

Knowledge base population is the task of taking a large body of unstructured text,
  and extracting from it a structured knowledge base.
Importantly, the knowledge base has a fixed schema of relations (e.g.,
  \ww{born in}, \ww{spouse of}), usually with associated type signatures.
These knowledge bases can then be used as a repository of knowledge for
  downstream applications -- albeit restricted to the schema of the knowledge base itself.
In fact, many downstream NLP applications do query large knowledge bases.
Prominent examples include
  question answering systems
    \cite{key:2001voorhees-trec},
  and semantic parsers
    \cite{key:1996zelle-semantics,key:2007zettlemoyer-semantics,key:2013kwiatkowski-semantics,key:2014berant-semantics}.

% Talk about relation extraction
Prior work in this area can be categorized into a number of approaches.
The most common of these are \textit{supervised} relation extractors
  \cite{key:2004doddington-ace,key:2005zhou-ace,key:2007surdeanu-ace},
  \textit{distantly supervised} relation extractors
  \cite{key:1999craven-distsup,key:2007wu-distsup,key:2009mintz-distsup,key:2011sun-kbp},
  and rule based systems
  \cite{key:1997soderland-kbp,key:2010grishman-kbp,key:2010chen-kbp}.


% this is a classification problem
Relation extraction can be naturally cast as a supervised classification problem.
A corpus of relation mentions is collected,
  and each mention $x$ is annotated
  with the relation $y$, if any, it expresses. The classifier's output
  is then aggregated to decide the relations between the two entities.

% costly to annotate
However, annotating supervised training data is generally
  expensive to perform at large scale.
Although resources such as Freebase or the TAC KBP knowledge base
  have on the order of millions of training tuples over
  entities it is not feasible to manually annotate the 
  corresponding mentions in the text.
This has led to the rise of \textit{distantly supervised} methods, which
  make use of this indirect supervision, but do not
  necessitate mention-level supervision.


\Fig{img/relation-extraction-setup}{0.5}{re-intro}{
  The relation extraction setup.
  For a pair of entities, we collect sentences which
    mention both entities.
  These sentences are then used to predict one or more relations between
    those entities.
  For instance, the  sentences containing both \textit{Barack Obama}
    and \textit{Hawaii} should support the \rel{state of birth} and
    \rel{state of residence} relation.
}

Traditional distant supervision makes the assumption that
  for every triple $(e_1, y, e_2)$ in a knowledge base between
  a subject $e_1$, a relation $y$, and an object $e_2$, every sentence
  containing mentions for $e_1$ and $e_2$ express the relation $y$.
% Example
For instance, taking \reffig{re-intro}, we would create a datum for each
  of the three sentences containing \ent{Barack Obama} and \ent{Hawaii}
  labeled with \rel{state of birth}, and likewise with
  \rel{state of residence}, creating 6 training examples overall.
Similarly, both sentences involving \textit{Barack Obama} and
  \textit{president} would be marked as expressing the \rel{title}
  relation.

% Downsides
While this allows us to leverage a large database effectively, it
  nonetheless makes a number of na\"{\i}ve assumptions.
First -- explicit in the formulation of the approach -- it assumes that every
  mention expresses some relation,
  and furthermore expresses the known relation(s).
For instance, the sentence \w{Obama visited Hawaii} would be erroneously
  treated as a positive example of the \rel{born in} relation.
Second, it implicitly assumes that our knowledge base is complete:
  entity mentions with no known relation are treated as negative examples.

% How these downsides are addressed
The first of these assumptions is addressed by
  multi-instance multi-label (MIML) learning, which puts an intermediate latent variable
  for each sentence-level prediction, that then has to predict the correct knowledge base
  triples \cite{key:2012surdeanu-mimlre}.
\newcite{key:2013min-incomplete} address the second assumption by extending the
  MIML model with additional latent variables, while \newcite{key:2013xu-filling}
  allow feedback from a coarse relation extractor to augment labels from the knowledge base.
These latter two approaches are compatible with but are 
  not implemented in this work.



% -- KBC
Lastly, there are approaches to inferring new facts in a knowledge base that do not
  make use of text at all, but rather aim to complete missing facts from the knowledge
  base based on patterns found from known facts.
For example, in the simplest case, if we know that a person usually lives in the state 
  of his employer, we can with high confidence predict a missing state of residence if
  we know a person's employer, and the employer's state of headquarters.

% Richard
\newcite{key:2013chen-completion} and \newcite{key:2013socher-completion}
  use Neural Tensor Networks for this task, predicting unseen relation triples in
  WordNet and Freebase.
This builds on a line of work by
  \newcite{key:2011bordes-completion} and
  \newcite{key:2012jenatton-completion} on knowledge base completion based on vector-space
  approaches.
% Universal schemas
\newcite{key:2012yao-schemas} and \newcite{key:2013riedel-schemas}
  present a related line of work, inferring new relations between
  Freebase entities via inference over both Freebase and
  OpenIE relations (see \refsec{related-openie}).
This work appeals to low-rank matrix factorization methods:
  in the simplest case, entity pairs form the rows of a matrix, and possible relations
  form the columns.
The entries of the matrix are 1 is a given relation holds between the given entity pair,
  and 0 otherwise.
This matrix is then factored into a low-rank approximation, in effect trying to compress
  the information in the knowledge base into a more concise representation.
Since this is an approximate factorization, some entries in the matrix which were previously
  0's now have a non-zero value -- these are taken to be new facts in the knowledge base.



%
% OPEN IE
%
\Section{related-openie}{Open Information Extraction}
% Talk about OpenIE
One approach to broad-domain knowledge extraction is \textit{open information extraction} (Open IE).
Traditional relation extraction settings usually specify a domain of relations they're
  interested in (e.g., place of birth, spouse, etc.), and usually place restrictions on the
  types of arguments extracted (e.g., only people are born places).
Open IE systems generalize this setting to the case where both the relation and the arguments
  are represented as plain text, and therefore can be entirely open-domain.
For example, in the sentence \ww{the president spoke to the Senate on Monday}, we might extract
  the following triples:

\begin{displayquote}
  (\ww{president}; \ww{spoke to}; \ww{Senate}) \\
  (\ww{president}; \ww{spoke on}; \ww{Monday})
\end{displayquote}


One line of work in this area are the early UW OpenIE systems: for example,
  TextRunner \cite{key:2007yates-textrunner} and
  ReVerb \cite{key:2011fader-reverb}.
In both of these cases, an emphasis is placed on \textit{speed} --
  token-based surface patterns are extracted that would correspond to
  open domain triples.
ReVerb, for instance, heuristically extracts relations centered around maximally expanded
  relations centered around a main verb, and attempts to find the best arguments
  for that relation subjected to certain constraints.
A confidence function is then trained over a set of features on the extractions, so that
  the system can provide calibrated confidence values.

With the introduction of fast dependency parsers,
  \cite{key:2010wu-openie} extracts triples from
  learned dependency patterns (in one mode), or POS-tagged surface features (for additional
  speed).
Building upon this, Ollie \cite{key:2012mausam-ollie} also learns patterns from dependency
  parses, but with the additional contributions of (1) allowing for extractions which are
  mediated by nouns or adjectives, not just verbs; and (2) considering context more carefully
  when extracting these triples.
Exemplar \cite{key:2013mesquita-exemplar} adapts the open IE framework to
  $n$-ary relationships similar to semantic role labeling, but without the
  associated expensive machinery of SRL systems.

In another line of work, The Never Ending Language Learning project (NELL) \cite{key:2010carlson-nell}
  iteratively learns more facts from the internet
  from a seed set of examples.
In the case of NELL, the ontology is open-domain but fixed, and the goal becomes to learn
  all entries in the domain.
For example, learning an extended hypernymy tree (\ww{Post University} is a University);
  but also more general relations (\ww{has acquired}, \ww{publication writes about}, etc).


Open IE has been shown to be useful in a number of downstream NLP tasks.
One line of work makes use of these open-domain triples to perform
  the fixed-schema relation extraction we discussed in \refsec{related-kbp}.
For example, \cite{key:2013soderland-kbp} constructs a mapping from open-domain
  to structured triples in only 3 hours of human effort, and achieves results
  which are competitive with (and higher precision than) custom-trained
  extractors.
\newcite{key:2010soderland-adapting} use ReVerb extractions to 
    enrich a domain-specific ontology.
Another line of work makes use of these triples in the context of knowledge
  base completion, as alluded to earlier \cite{key:2012yao-schemas,key:2013riedel-schemas}.
In this case, we are learning a sort of soft mapping from Open IE triples to
  structured triples implicitly in the matrix factorization.
In a related vein, OpenIE has been used directly for learning entailment patterns.
For example, \newcite{key:2010-schoenmackers-horn} and \newcite{key:2011berant-entailment}
  learn valid open-domain entailment patterns from OpenIE extractions.
\newcite{key:2011berant-entailment}, for instance, presents a method for learning
  that the relation \ww{be epidemic in} should entail \ww{common in} should entail
  \ww{occur in}.

Another prominent use-case for open domain triples is
  question answering and search \cite{key:2014fader-openqa,key:2011etzioni-nature}. 
In \newcite{key:2014fader-openqa}, a set of question paraphrases is mined from a large
  question answering corpus; these paraphrases are then applied to a new question until
  it matches a known Open IE relation.
In each of these cases, the concise extractions provided by open IE allow
  for efficient symbolic methods for entailment, such as Markov logic
  networks or matrix factorization.


%
% COMMON SENSE
%
\Section{related-commonsense}{Common Sense Reasoning}

% -- GOFAI
% (intro)
The goal of tackling common-sense reasoning is by no means novel in
  itself either.
Among others, work by \newcite{key:1980reiter-logic} and \newcite{key:1980mccarthy-circumscription}
  attempts to reason about the truth of a consequent in the absence of strict logical entailment.
Reiter introduces a notion of \textit{default reasoning}, where facts and entailments can be
  taken as true unless there is direct evidence to the contrary.
For example, \ww{birds fly} ($\textrm{bird}(x) \Rightarrow \textrm{flies}(x)$) is a statement which
  is clearly false in the strictest sense -- penguins and ostriches don't fly -- but nonetheless
  a statement that we would consider true.
Reiter argues that common-sense statements of this sort should be considered true by default in
  a formal reasoning engine, unless there is explicit evidence against them.
McCarthy argues a related point on circumscription: that in the absence of evidence to the contrary,
  we should assume that all prerequisites for an action are met.
For instance, if someone claims to have a rowboat, we should assume that they have oars, the oars
  fit into the rowlocks, the boat is without holes, etc.; unless explicitly told otherwise.
In another line of work, \newcite{key:1989pearl-probabilistic} presents a framework for
  assigning confidences to inferences which can be reasonably assumed.

More recently, work by \newcite{key:2002schubert-commonsense} and
  \newcite{key:2009durme-commonsense} approach common sense reasoning
  with \textit{episodic logic}.
Episodic logic is a first-order logic which focuses primarily on time-bounded
  situations (events and states), rather than the time-insensitive propositions
  of conventional first-order logic.
For example, a sentence \ww{John kicked Pluto} could be interpreted as:

\begin{equation*}
\exists e1 : [ e1 ~ \textrm{before Now1} ] \left[ [ \textrm{John kick Pluto} ] **~ e1 \right]
\end{equation*}

That is, there is an event $e1$ that happened before now (the reference time), and the various
  the sentence \ww{John kick Pluto} characterizes or fully describes $e1$.
The operator $**$ has an analogous operator $*$ for cases where the statement (or formula)
  is \textit{true} in the event, but does not fully characterize it
For example:

\begin{equation*}
\exists e1 : [ e1 ~ \textrm{before Now1} ] \left[ [ \textrm{John has a leg} ] **~ e1 \right]
\end{equation*}


Lastly, efforts like MIT's ConceptNet \cite{key:2011tandon-conceptnet} and Cyc \cite{key:1995lenat-cyc}
  aim to create a comprehensive, unified knowledge base of common-sense facts.
For instance, ConceptNet has facts like (\ww{learn}; \ww{motivated by goal}; \ww{knowledge}).
These facts are collected from a number of sources, both automatic (e.g., Open IE) and hand-curated
  (e.g., WordNet).
Cyc catalogs in a structured way facts like: There exists a female animal for every Chordate
  which is described by the predicate \textit{mother}.
All of these facts are hand-curated.


%
% RTE
%
\Section{related-rte}{Textual Entailment}

% -- RTE
This dissertation is in many ways to work on 
  recognizing textual entailment (RTE) -- e.g., 
  \newcite{key:2010-schoenmackers-horn}, \newcite{key:2011berant-entailment}.
Textual Entailment is the task of determining if a given premise sentence
  entails a given hypothesis.
That is, if without additional context, a human would infer that the hypothesis
  is true if the premise is true.
For instance:

\begin{displayquote}
  \ww{I drove up to San Francisco yesterday} \\
  \ww{I was in a car yesterday}
\end{displayquote}

Although the definition of entailment is always a bit fuzzy -- what if I drove a train
  up to SF, or perhaps a boat? -- nonetheless a reasonable person would assume that if
  you drove somewhere you were in a car.
This sort of reasoning is similar to the goal of this dissertation: given premises, infer
  valid hypothesis to claim as true.
However, in RTE the premise set tends to be very small (1 or 2 premises), and the domain
  tends to have less of a focus on common-sense or broad domain facts.


Work by \newcite{key:2013lewis-entailment}
  approach entailment by constructing a CCG parse of the query,
  while mapping questions which are paraphrases of each other to the
  same logical form using distributional relation clustering.
\newcite{key:2008hickl-rte} approaches the problem by segmenting the premise and
  the hypothesis into a set of \textit{discourse commitments} that a reader would
  accept upon reading the sentence.
This can then be used to determine whether the commitments in the premise match those in
  the hypothesis.
For example, the following premise, taken from the RTE-3 challenge:

\begin{displayquote}
``The Extra Girl'' (1923) is a story of a small−town girl, Sue Graham (played by Mabel Normand) who comes to
  Hollywood to be in the pictures. 
This Mabel Normand vehicle, produced by Mack Sennett, followed earlier
  films about the film industry and also paved the way for later films about Hollywood, such as King Vidor’s
  "Show People" (1928).
\end{displayquote}

would yield the commitments:

\begin{displayquote}
T1. "The Extra Girl" [took place in] 1923. \\
T2. "The Extra Girl" is a story of a small−town girl. \\
T3. "The Extra Girl" is a story of Sue Graham. \\
T4. Sue Graham is a small−town girl. \\
T5. Sue Graham [was] played by Mabel Normand. \\
$\dots$ \\
T10. ``The Extra Girl'' [was] produced by Mack Sennett. \\
T11. Mack Sennett is a producer.
\end{displayquote}

This could then be used to easily justify the hypothesis:
  \ww{``The Extra Girl'' was produced by Sennett}.


% Natural Logic
This dissertation focuses a fair bit on natural logic.
The primary application of natural logic in prior work has been
  for textual entailment and related fields.
For example, work by MacCartney 
  \cite{key:2007maccartney-natlog,key:2008maccartney-natlog,key:2009maccartney-natlog,key:2009maccartney-natlog}
  has applied natural logic to the RTE challenges described above.
This work approaches the entailment task by inducing an alignment between the premise and the
  hypothesis, classifying entailed pairs into natural logic relations, and then inferring
  from this alignment and the semantics of natural logic what the correct entailment
  relation is between the sentences.
This line of work has been extended since in, e.g., \newcite{key:2012watanabe-natlog}.
More background on natural logic is given in \refchp{natlog}.


More recently, work by \newcite{key:2013bowman-natlog}
  (and \newcite{key:2015bowman-natlog} has explored using vector spaces
  as a meaning representation for natural language semantics.
Entailment was chosen as difficult, comprehensive evaluation of natural language
  understanding, and since natural logic is inherently lexical, it was chosen
  as the formalism against which to test these models.
Results from these papers show that, indeed, appropriately composing
  vector-space representations of words can yield systems which capture a
  notion of entailment with high accuracy.

Since the discontinuation of the RTE challenges, there have been a number of
  new efforts in creating datasets for textual entailment against which
  systems can be trained.
For example, \newcite{key:2014marelli-sick} produce a dataset of entailment
  pairs from image captions which is an order of magnitude larger than the RTE
  datasets.
Subsequently, \newcite{key:2015bowman-natlog} created a very large corpus
  of human-annotated entailment pairs, again from image captions.
This dataset has since been used as the de-facto large-scale dataset on which
  to evaluate entailment models, and most prominently neural sequence models for
  entailment.
For example, \newcite{key:2015rocktaschel-snli} construct a recurrent neural network
  with attention for entailment; \newcite{key:2016long-snli} and other
  follow up on that work with a new attention mechanism.



%
% QA
%
\Section{related-rte}{Question Answering}
Many parts of this dissertation are reminiscent of question answering tasks.
Related work on question answering can be segmented into two broad categories:
Early work focused on question answering directly over the text of a
  source corpus.
For example, taking as input a Wikipedia article, and using that text to answer
  questions about the article's subject.
Later work, in contrast, focuses on using structured knowledge bases for question
  answering with an emphasis on moving in the direction of 
  complicated compositional queries.

% Direct over text Q/A
A representative task in the first line of work -- making direct use of text
  for question answering -- is the TREC Q/A competitions 
  \cite{key:2001voorhees-trec,key:2006voorhees-trec,key:2007dang-trec,key:2008voorhees-trec}.
An early example of such a system is Textract \cite{key:1999srihari-trec}, based around 
  a pipeline of information extraction components.
The system would take a question (e.g., \ww{who is the president of the United States?})
  and convert it into a target named entity type (i.e., PERSON), and a set of keywords
  (i.e., \ww{president}, \ww{United States}).
The keywords are then fed into a query expansion module to expand the domain of candidate
  premises returned, and the resulting sentences are scanned for an entity of the right
  named entity type based on some heuristics.

This can be cast as a simplistic version of the two-stage Q/A approach of running information
  retrieval (IR), and then classifying the results into whether they are an answer to the
  questions.
Subsequent work has expanded on this sort of IR+classify approach to, e.g., use language
  modeling to assist in reranking results \cite{key:2006chen-trec}, incorporate
  more sophisticated methods for learning to rank \cite{key:2007cao-trec},
  incorporating additional knowledge sources such as Wikipedia \cite{key:2004ahn-trec},
  etc.

The other main line of work uses structured knowledge bases for question answering.
The main line of work here is in the tradition of semantic parsing
  \cite{key:2005kate-semantics,key:2005zettlemoyer-semantics,key:2011liang-semantics}.
An input query is parsed into a structured logical representation; this representation
  can then be run against a knowledge base to retrieve the answer to the query.
For example, \newcite{key:2005zettlemoyer-semantics} and \newcite{key:2011liang-semantics} 
  parse complex, compositional geological
  queries into a logical form that can be evaluated against a database of geological facts.
A complex sentence like:

\begin{displayquote}
  \ww{What states border the state that borders the most states?}
\end{displayquote}

Would be parsed into a logical form (effectively, a structured query) like:
\begin{equation*}
\lambda x . state(x) \land borders(x, \argmax( \lambda y . state(y), \lambda y . count (\lambda z . state(x) \land borders(y, z))))
\end{equation*}

\newcite{key:2005zettlemoyer-semantics} approaches the problem by learning a synchronous
  combinatory categorical grammar (CCG) \cite{key:2004bos-ccg} 
  parse over lexical items and logical form fragments,
  given a fixed grammar.
Until convergence, the algorithm performs the following two steps:
\begin{itemize}
  \item Expand the set of lexical mappings (e.g., \ww{state} means $\lambda x . state(x)$) 
        known to the algorithm. 
        This is done by performing the following on each example in the training set:
        (1) Generate all possible lexical mappings, given the logical form and surface
            form of the example.
        (2) Parse the example, given the current learned parameters.
        (3) Add to the set of lexical mappings all lexical mappings in the parsed
            logical form.
  \item Learn a new parameter vector for the parser, given the lexicon so far and the
        training set.
\end{itemize}

Work by \newcite{key:2011liang-semantics} builds on this approach by removing the need for
  an annotated logical form, and instead running a two-step EM-like learning algorithm
  (on a fixed CFG rather than CCG grammar):
First, a set of possible parses for each example are induced given the current parameter vector.
The parses which evaluate to the correct \textit{answer} are treated as correct and renormalized
  proportional to their original score.
This is then used as a training signal to learn a new parameter vector.

This same type of approach can be applied more broadly to more practical types of questions.
For example, \newcite{key:2013berant-sempre} apply a similar approach to answering
  questions about entities in Freebase -- a popular large knowledge base containing facts
  about people, places, organizations, etc.
In this case, utterances are factoid-style Q/A questions similar to the TREC questions;
  e.g., \ww{what college did Obama go to?}.
This is then parsed into a logical form, and executed against the Freebase knowledge graph
  to produce the desired answer: Columbia and Harvard.


Subsequent work has also applied semantic parsing to even more broad-domain representations.
For example, work by \newcite{key:2009artzi-amr} uses semantic parsing methods to parse into
  the Abstract Meaning Representation \cite{key:2013banarescu-amr} -- a 
  broad-domain structured meaning representation.
New methods (e.g., \newcite{key:2014bordes-qa}) have applied neural methods
  to embed knowledge base fragments and natural language questions in vector spaces,
   and answer questions based on neural network methods.

% -- Logic-based QA
Lastly, there has been work on incorporating more structured logical
  reasoning for question answering.
The COGEX system \cite{key:2003moldovan-trec} incorporates a theorem
  prover into a QA system, boosting overall performance.
Similarly, Watson \cite{key:2010ferrucci-watson} incorporates
  logical reasoning components.
In a new spin on logical methods for question-answering,
  \newcite{key:2015hixon-aristo} proposes
  a dialog system to augment a knowledge graph used for answering science exam
  questions.
This is in a sense an oracle measure, where a human is consulted while answering
  the question to substantially improve accuracy.
Furthermore, they show that their additional extractions help
  answer questions other than the one the dialog was collected for;
  that is, human-in-the-loop Q/A improves accuracy even on unseen questions.

