\prefacesection{Acknowledgments}

First and foremost, I'd like to thank my advisor, Chris Manning, who is the reason this dissertation
  exists.
Throughout my Ph.D., he has been amazingly supportive of whatever path I want to take
  in research (and, for that matter, life), while nonetheless caring enough to make sure the things
  I was doing week-to-week were going to lead me in the right direction.
Chris: for good advice, for everything you've taught me, and most of all for being a steady ally
  throughout my Ph.D. -- thank you.

I'd also like to thank a few of the many other professors and advisors who have helped shape my research.
Percy Liang: I'm certain I wouldn't be where I am today without your patience, teaching, and kindness; both
  as a mentor in my undergrad and as a professor at Stanford.
Much of what I know about NLP and how to do research I've learned from you.
Dan Klein: Thank you for introducing me to AI and NLP, and for being welcoming during my undergrad.
Dan Jurafsky: Thank you for advising me for my first research paper at Stanford, for teaching me
  to think before I code (painful as it is), and for being enthusiastic and supportive throughout
  my Ph.D.
Thomas Icard: Thank you for numerous discussions about natural logic that never failed to make
  my brain hurt by the end.

Research is never done in a vacuum; I'm deeply indebted to my colleagues and co-authors in the
  NLP group.
I'm fortunate to count all of them as friends.
From my early foray into semantic parsing,
  Jakob Uszkoreit: for a summer glimpse into the world of industry, and for reassurance
  that people really do care about time.
Angel Chang: for tea time and stimulating discussions, technical
  and otherwise.
From my alternate life building the kinds of knowledge bases this dissertation happens
  to argue against,
Arun Chaganty: for being my co-conspirator in many things, not the least of which was KBP.
Julie Tibshirani and Jean Wu: for being lively officemates and invaluable help with KBP.
Melvin Premkumar: for suffering through yet-another rewrite of the code, and for
  your help with OpenIE.
From my life pretending to be a linguist and a logician, which is much of what I do in this
  dissertation,
Sam Bowman: for invaluable discussions about natural logic, and being a real linguist I can learn from.
I'd also like to thank Neha Nayak and Keenon Werling for being a lot of fun to mentor,
  and who have both taught me a lot as well.
And of course I'm grateful to all the members of the NLP group, both the more senior and more junior.
I've learned a great deal from all of you.

As graduate students we're not supposed to have much of a social life, but luckily I've
  found friends who make sure I don't always do what I'm supposed to.
Chinmay Kulkarni: for always being up for an adventure.
Ankita Kejriwal: for being a thoughtful friend in every sense of the word.
Irene Kaplow and Dave Held: for always being bubbly and cheerful, and for being wonderful study partners
  back when we were all clueless first years.
Joe Zimmerman: for interesting conversations about crypto, deep learning, and memes.
The AI and Graphics labs: for making the second and third floors of Gates welcoming.

Speaking of folks in the Graphics lab, I want to especially thank Sharon Lin.
For being helpful when life is hard, for being patient when life is busy, and most importantly
  for sharing in (and in many cases, creating) the joy when life is good.
I have no doubt that my time at Stanford would have been very different without you.

I started by thanking my academic ``parents,'' but truth is they would have never
  even met me were it not for family, and most importantly my parents: Aniko Solyom and George Angeli.
So much of your lives have been in pursuit of better life for my brother and I; it's a debt I cannot
  repay.
From weekend chess tournaments, to helping with science fair projects and my early flirtation with research,
  to supporting me through undergrad.
Thank you.
