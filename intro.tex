
%
% Philosophical BS
%
At its core, machine-learning-driven natural language processing aims to imitate human intelligence by
  observing a human perform a given task repeatedly and training from this data.
For example, in order to train a system to recognize whether a given word is the name of a person,
  we would first collect a large set of words, labeled as either people or not.
A system would then take this data, and learn a \textit{model} that can then predict, on unseen
  words, whether it is a person or not.
The great advantage of this framework is that it frees us from having to have a deep understanding of
  the underlying process by which humans perform the target task, instead allowing us to observe
  examples and use this to learn to replicate the task.
In fact, this has been responsible for much of the progress in natural language processing in
  the first two decades of the new millennium.
Many of the core NLP tasks (named entity recognition, part of speech tagging, parsing, etc.)
  can now be done with high accuracy,
  and many of the higher-level tasks (relation extraction, sentiment analysis, question-answering, etc.)
  have matured to the point of being useful as off-the-shelf components both for academia and industry.

With these advances, I believe that we should turn back to a relatively neglected
  topic: how do we begin to create programs that exhibit \textit{general purpose} intelligence?
In many ways, data-driven natural language processing systems are idiot savants:
  these systems perform at impressive accuracies at very narrow tasks -- the tasks they were trained
  to replicate -- but are incapable of either generalizing across tasks, or performing complex
  common-sense inferences.
For example, the following is a list of some questions which are trivial for humans to answer, 
  but are very difficult
  for a trained system without either (1) very specific, narrow, and deep training data, or (2) a very large
  amount of general-purpose background knowledge:

\begin{displayquote}
  \ww{\textbf{I ate a bowl of soup, and put the bowl in the bathtub. Did it float?}} \\
  Answering this question correctly requires not only a fairly complex bit of inference, but also
    a large amount of varied background knowledge: a bowl is concave, empty concave things float,
    if you eat soup the bowl becomes empty, bathtubs are full of water, etc.
\end{displayquote}

\begin{displayquote}
  \ww{\textbf{I left water in the freezer; what happened to it?}} \\
  Here again, we need to know that freezers are cold (below freezing), that water turns to ice
    when it's below freezing, and that water turning to ice is more informative than the other things
    that also ``happen'' to it, such as it getting cold, or getting dark, or no longer sloshing, etc.
\end{displayquote}

\begin{displayquote}
  \ww{\textbf{The Congressman resigned to go back to governing his hometown. What is his new title?}} \\
  To correctly answer ``mayor,'' we would have to know that if someone resigns from a title, he no longer
    holds it, and that a mayor governs a town.
  Also, that a hometown is a city or other entity with a mayor -- unlike, say, homework or downtown.
\end{displayquote}

%
% Thesis Intro
%

A central tenet of this dissertation is that, if we're in pursuit of general intelligence, we should be
  aiming to answer these sorts of questions not by collecting narrow-domain deep training sets, but rather
  by developing techniques to collect large amounts of common-sense information at scale.
For example, we can teach a computer to play Chess or Go at super-human levels, but an algorithm
  for playing Go cannot play Chess.
We have collected (or generated) deep training data for one of these games, but it is so narrow domain
  that the data is near worthless at the other game.
In the same way, we can train statistical models to predict the sentiment of a movie review, but these models
  do not do well on any task other than the very narrow task it was trained for.
Going forward, I believe a key property of broad domain intelligent systems will be the ability to 
  leverage vast amounts of \textit{common sense} knowledge to perform tasks.
This is, after all, the same common-sense information we leverage to solve these tasks ourselves.
But, this is also the kind of knowledge that is hard to collect a training set for.
Whereas as few million movie reviews should be sufficient to get a good sense for what sort of reviews
  are positive versus negative, it's certainly not true that a few million common-sense facts do
  not even make a dent on the number of facts we as humans know, and require to perform the reasoning
  we do.

An additional axiom of the dissertation is that the most promising way to collect this sort of 
  common-sense knowledge is from text.
This is not not an uncontroversial axiom: much of what we know about the world we know from
  vision, or interactions with physical objects.
Colors, shapes, and sizes of things for example are far more naturally extracted from visual data;
  the laws of physics are much more easily learned through interaction with the world.

Nonetheless, natural language is the de-facto standard for storing and transmitting 
  information, and has the key advantages of being plentiful and convenient to work with.
Textual data stores information about people and places (e.g., \w{Obama was born in 
  Hawaii}), facts about science and engineering (e.g., \w{Ice is frozen water}), 
  or simply common-sense facts (e.g., \w{Some mushrooms are poisonous}). 
It's also important to remember that with the internet, we have unprecedented access to a 
  huge -- and growing -- amount of text.

The most natural solution to this problem of common-sense knowledge is to collect large 
  manually-curated (or semi-curated) databases of these general purpose facts.
Freebase \cite{key:2008bollacker-freebase} is a popular example of one
  such database; Cyc \cite{key:1995lenat-cyc} is another famous hand-curated
  knowledge base.
However, these manually created knowledge bases are both woefully incomplete 
  and quickly become outdated.
In medicine, half of the medical school curriculum becomes obsolete within 5 
  years of graduation,\footnote{
    \url{http://uvamagazine.org/articles/adjusting\_the\_prescription/}
  }
  requiring constant updating.
  MEDLINE counts 800,000 biomedical articles published in 2013 alone.\footnote{
    \url{https://www.nlm.nih.gov/bsd/medline\_cit\_counts\_yr\_pub.html}
  }
In academia, studies show that up to 90\% of papers are never 
  cited, suggesting that many are never read.
Moreover, it's often hard to represent in a knowledge base facts which are easy to 
  represent in language.
For instance: ``\ww{A graduated cylinder is the best tool to measure the volume of a
  liquid},'' or ``\ww{strawberries are probably one of the best foods}.''

This dissertation presents an alternative:
  instead of storing knowledge in knowledge bases, let's continue to store knowledge
  directly in text.
This allows us to make use of our large corpus of unstructured text directly, with
  perhaps minimal syntactic annotation, to reason about true and false facts.
Storing information becomes as easy as storing text.
But, this comes at a cost.
The key challenge in this approach is the ability to use a large 
  corpus of plain text to query facts and answer questions which are not verbatim 
  expressed in the text.
For example, a statement \ww{the cat ate a mouse} should support even lexically 
  dissimilar queries like \ww{carnivores eat animals} and reject logically 
  contradicted queries (like \ww{no carnivores eat animals}).
Or, it may be difficult to find the relevant nugget of information from a long and syntactically
  complex sentence from, e.g., Wikipedia.
Therefore, the technical focus of this dissertation will be on how we run soft logical
  inference from a very large set of candidate premises (our source corpus) to either
  prove or disprove a candidate fact whose truth we do not know.

A natural formalism for addressing this challenge is \textit{natural logic} -- a proof 
  theory over the syntax of natural language.
The logic offers computational efficiency and eliminates the need for semantic 
  parsing and domain-specific meaning representations, while still warranting most 
  common language inferences (e.g., negation). 
Furthermore, the inferences warranted by the logic tend to be 
  the same inferences that are cognitively easy for humans -- that is, 
  the inferences humans assume a reader will effortlessly make.

This dissertation explores how to leverage natural logic as a formalism for
  extracting knowledge not only when it is verbatim written in text, but also when
  it is only \textit{implied} by some statement in the text and we must perform
  a large scale inference over a large set of candidate premises to find the right one
  (if any).
In the subsequent chapters, we will review the theory behind natural logic
  (\refchp{natlog}), and then describe a system to
  (1) extract common-sense knowledge from a large corpus of unannotated text via 
    a search procedure over a soft relaxation of natural logic;
  (2) simplify complex syntactic structures into maximally informative atomic
    statements, and
  (3) incorporate an entailment classifier into this search to serve as an
    informed backoff.


%
% NaturalLI
%

In \refchp{naturalli} we introduce our general framework for inferring the truth or 
  falsehood of common-sense facts from a very large knowledge base 
  of statements about the world.
For example, if a premise \ww{the cat ate a mouse} is present in the knowledge 
  base, we should conclude that a hypothesis \ww{no carnivores eat animals} is false.
The system constructs a search problem for each queried hypothesis over relaxed 
  natural logic inferences: the surface form of the hypothesis is allowed to mutate 
  until it matches one of the facts in the knowledge base.
These mutations correspond to steps in a natural logic proof; a learned cost for each 
  mutation corresponds to the system's confidence that the mutation is indeed 
  logically valid (e.g., mutating to a hypernym has low cost, whereas 
  nearest neighbors in vector space has high cost).
This amounts to a high-performance fuzzy theorem prover over an arbitrarily 
  large premise set, where the extent to which particular logical mutations
  are correct or incorrect can be learned from a training set of known correct
  and incorrect facts.


An illustration of a search from the query \ww{no carnivores eat animals} 
  is given below, with the appropriate natural logic relation annotated 
  along the edges.
The mutation from \ww{no} to \ww{the} negates the sentence (\negate); the mutations
  from \ww{carnivore} to \ww{cat}, the introduction of \ww{an}, and the mutation from
  \ww{animal} to \ww{mouse} all preserve negation (\reverse).
Therefore, the premise negates the query fact:


\vspace{1cm}
\begin{center}
\teaserSearch
\end{center}
\vspace{1cm}


This framing of the problem has a number of advantages.
Prior work in \textit{textual entailment} -- the task of determining if a premise sentence logically entails the
  truth of a hypothesis -- has traditionally only dealt with very small (1 or 2 sentence) premise sets.
In contrast, this approach scales to arbitrarily large premise sets.
Prior work in large-scale inference -- for example approaches making use of large graphical
  models (e.g., Markov Logic Networks \cite{key:2006richardson-mln}) -- tend to be
  computationally inefficient as the size of the inference problem grows.
This approach becomes more efficient the larger the premise set grows, since we can run
  a shallower search in expectation.
From the other direction, unlike information retrieval approaches, we remain 
  sensitive to a notion of entailment rather than simply similarity -- for example, 
  we can detect false facts in addition to true ones.
In an empirical evaluation, we show that we can recover 50\% of common sense facts 
  at 90\% precision -- 4x the recall of directly querying 
  a very large source corpus of 270 million premises.



%
% OpenIE
%
Although this approach works well in cases where the the source corpus is composed
  of simple atomic facts, most useful facts in the wild are embedded as small parts
  of more complex sentences.
This is, in fact, relevant not only as a
  subcomponent in our reasoning engine (it will allow us to digest complex
  sentences from real-world data sources and segment them into atomic facts)
  but also as a standalone application.
A common motif in information extraction in general is the value in 
  converting such complete sentences into a set of atomic propositions.

\refchp{openie} describes our system to extract atomic propositions (e.g., 
  \ww{Obama was born in Hawaii}) from longer, more syntactically difficult 
  sentences (e.g., \ww{Born in Hawaii, Obama attended Columbia}) by recursively 
  segmenting a dependency tree into a set of self-contained clauses expressing 
  atomic propositions.
This segmentation is done by defining a breadth-first search on a dependency tree,
  where at each arc a decision is made among a set of actions determining whether
  to split off the subordinate clause, and if so how that split should occur.
These clauses are then maximally shortened to yield propositions which are
  logically entailed by the original sentence, and also maximally concise.
For instance, the statement 
  \ww{anchovies were an ideal topping for Italian sailors} 
  yields \ww{anchovies are a topping.} 

For example, the figure below shows the system segmenting the sentence
  \ww{born in a small town, she took the midnight train going anywhere} into
  two ``clauses,'' and then shortening each clause to obtain maximally informative
  short propositions.
The left ``clause'' is the entire sentence.
In that context, the main message we are trying to convey is that a girl (``\textit{she}'') 
  took the midnight train.
Her birthplace, the destination of the train, etc. are supplemental bits of information.
Therefore, we are justified in stripping off these additional modifiers, and arriving
  at a maximally concise utterance (i.e., dependency tree fragment) \ww{she took midnight train}.
However, we also extract \ww{she [was] born in a small town} as a separate clause.
Now, her birthplace is the central theme of the clause, and the most we can strip off is the qualifier
  \ww{small} on \ww{town}:

\begin{center}
\scalebox{0.85}{
\begin{tabular}{cc}
\hspace{-3ex}
  % Full tree
  \begin{dependency}[text only label, label style={above}]
    \begin{deptext}[column sep=-0.00cm]
      Born \& in \& a \& small \& town \&[-1ex] , \& she \& took \& the \&
        midnight \& train \& going \& anywhere \&[-1ex] . \\
    \end{deptext}
    \depedge[edge unit distance=1.25ex]{1}{5}{prep\_in}
    \depedge[edge unit distance=1.0ex]{5}{4}{amod}
    \depedge[edge unit distance=1.4ex]{5}{3}{det}
    \depedge[edge unit distance=1.0ex, edge style={darkred!60!black,thick,densely dotted}]{8}{1}{\textbf{\darkred{vmod}}}
    \depedge[edge unit distance=2.0ex, edge style={blue!60!black,thick}]{8}{7}{\darkblue{nsubj}}
    \depedge[edge unit distance=1.75ex]{8}{11}{dobj}
    \depedge[edge unit distance=1.0ex]{11}{10}{nn}
    \depedge[edge unit distance=1.4ex]{11}{9}{det}
    \depedge[edge unit distance=1.0ex]{11}{12}{vmod}
    \depedge[edge unit distance=1.0ex]{12}{13}{dobj}
  \end{dependency}
  &
  % Just Clause
  \begin{dependency}[text only label, label style={above}]
    \begin{deptext}[column sep=-0.05cm]
      she \& Born \& in \& a \& small \& town \\
    \end{deptext}
    \depedge[edge unit distance=1.25ex]{2}{6}{prep\_in}
    \depedge[edge unit distance=1.0ex]{6}{5}{amod}
    \depedge[edge unit distance=1.4ex]{6}{4}{det}
    \depedge[edge unit distance=2.0ex, edge style={blue!60!black,thick}]{2}{1}{\darkblue{nsubj}}
  \end{dependency}
  \\
  \textbf{\small{(input)}} & \textbf{\small{(extracted clause)}} \\
  % arrows
  $\downarrow$ & $\downarrow$ \\
  % entailments
  \begin{tabular}{l}
    \ww{\small{Born in a small town, she took the midnight train going anywhere}} \\
    \ww{\small{Born in a small town, she took the midnight train}} 
    \ww{\small{\textbf{she took midnight train}}}  \\
    $\dots$ \\
  \end{tabular} &
  \begin{tabular}{l}
    \ww{\small{\textbf{she Born in small town}}} \\
    \ww{\small{she Born in a town}} \\
    \ww{\small{\textbf{she Born in town}}} \\
  \end{tabular}
\end{tabular}
}
\end{center}



In addition to being a component in our reasoning engine,
  we can directly use this method for Open Information Extraction 
  (Open IE; see \refsec{related-openie}) -- a flavor of relation 
  extraction where the relation, subject, and 
  object are all allowed to be \textit{open domain} plain-text strings.
On a NIST-run knowledge base population task, we show that this system 
  achieves good results, outperforming all prior work applying OpenIE
  to structured relation extraction by 3 F$_1$.
Despite not being developed for this 
  task, our system achieves a score halfway between the median and top 
  performing system, outperforming multiple purpose-built systems.


%
% Aristo
%
Lastly, a key property of natural logic is its ability to interface nicely with 
  statistical models which featurize the surface form of a sentence.
This allows us to construct models which capture the broad coverage and high recall
  of shallow featurized models, while nonetheless maintaining much of the nuanced logical
  reasoning that natural logic provides.
This is in contrast to most earlier work using structured logic for question answering
  tasks, which struggle to get reasonable recall.
For example, \newcite{key:2005bos-rte} observe that, in the RTE task of determining whether a
  sentence entails another sentence, strict logical methods obtain only 5.8\% recall.


\refchp{qa} outlines a method for combining these two signals -- a shallow
  featurized classifier and our natural logic engine -- elegantly in
  a single unified model.
We continue to view our problem as a search problem, where we mutate the query fact
  in ways that are warranted (or approximately warranted) by natural logic until we
  hit a premise fact.
However, since each of the intermediate states in this search is in itself a natural
  language sentence, we can featurize these intermediate states and run a classifier
  to predict whether there is a premise which is \textit{likely} to entail (or contradict) that state.
Since the intermediate state entails (or contradicts) the query fact, we can infer
  from transitivity that the same premise entails / contradicts the query fact.
This can be thought of as an evaluation function in the same way that a gameplaying agent
  uses an evaluation function to assess the game state.
If the search problem reaches a terminal state (a known premise in our case, analogous to, e.g., a checkmate
  in Chess) then the evaluation function is not needed.
However, if no terminal state is found, it is nonetheless useful to have an estimate for how close we
  are to a terminal state.

The evaluation function is carefully designed to be easy to update during the search
  process.
It makes use of a set of alignment features -- e.g., as in the figure below -- such that
  as we mutate the premise (\ww{rain and snow are types of precipitation}) we can match
  fewer or more alignments.
Weights over these features are learned, presumably such that more valid alignments
  is positively correlated with entailment.
Note that even in the below example, despite the premise and the hypothesis being substantially
  similar they are not technically an entailment pair, and therefore the evaluation function is crucial
  to produce an entailment relation.
Moreover, note that as the search progresses (e.g., \ww{types} is replaced by its synonym \ww{forms})
  the confidence of entailment will improve.

\begin{center}
\begin{tikzpicture}
\tikzset{
  arm angleA/.initial={0},
  arm angleB/.initial={0},
  arm lengthA/.initial={0mm},
  arm lengthB/.initial={0mm},
  arm length/.style={%
    arm lengthA=#1,
    arm lengthB=#1,
  },
  arm/.style={
    to path={%
      (\tikztostart) -- ++(\pgfkeysvalueof{/tikz/arm angleA}:\pgfkeysvalueof{/tikz/arm lengthA}) -- ($(\tikztotarget)+(\pgfkeysvalueof{/tikz/arm angleB}:\pgfkeysvalueof{/tikz/arm lengthB})$) -- (\tikztotarget)
    }
  },
}
\matrix[column sep=-0.0em,row sep=0.75cm,matrix of nodes,row 2/.style={coordinate}] (m) {
% First sentence
\rnode{Forms}{$^p$} & \hnode{of} & \rnode{precipitation}{$^p$} & \bnode{include}{$^p$} & \rnode{rain}{$^p$} & \hnode{and} & \rnode{sleet}{$^p$} \\
\\
% Second sentence
\rnode{Rain}{$^c$} & \hnode{and} & \rnode{snow}{$^c$} & \hnode{are} & \rnode{types}{$^c$} & \hnode{of} & \rnode{precipitation}{$^c$} \\
};
\begin{scope}[every path/.style={line width=1pt,white,double=black},every to/.style={arm}, arm angleA=-90, arm angleB=90, arm length=5mm]
 \draw [draw=green] (rain$^p$) to (Rain$^c$);
 \draw [draw=red] (Forms$^p$) to (types$^c$);
 \draw [draw=green] (precipitation$^p$) to (precipitation$^c$);
 \draw [draw=red  ] (sleet$^p$) to (snow$^c$);
\end{scope}
\end{tikzpicture}
\end{center}


We evaluate this complete system on 4\nth\ grade science exam questions, and show that we 
  outperform prior work, a strong information retrieval baseline, and a 
  standalone version of the evaluation function.
%We can achieve a final score of 74\% on our practice test, and 67\% on unseen 
%  test questions.



%
% Going forward
%
Together, these contributions form a powerful reasoning engine for inferring open-domain
  knowledge from very large premise sets.
Unlike traditional IR approaches, or shallow classification methods, the system maintains
  a notion of logical validity (e.g., proper handling of negation);
  unlike structured logical methods, the system is high-recall and robust to
  real-world language and fuzzy inferences.
From here, the foundation is laid to \textit{leverage} this sort of knowledge in a
  general way, in pursuit of systems that exhibit broad-domain intelligence.
Returning to the examples from the beginning of the section, if we are faced with a question like:

\begin{displayquote}
  \ww{I ate a bowl of soup, and put the bowl in the bathtub. Did it float?}
\end{displayquote}

\noindent We have a method for determining from a large unlabeled corpora of text
  that a bowl is concave, empty concave things float, and so forth.
We do this by exploiting three key insights: \textbf{first}, that we can run statistical natural logic proofs
  over a very large natural language premise set by casting the task as a search problem, producing
  the most efficient open-domain reasoning engine over such a large number of premises (published as
  \newcite{key:2014angeli-naturalli}).
\textbf{Second}, we can segment long, syntactically complex sentences into simple atomic utterances that are
  convenient for this natural logic search problem (published as \newcite{key:2015angeli-openie}).
\textbf{Third}, we can augment the search with an evaluation function, which ensures high recall in retreiving
  the truth of these facts (published as \newcite{key:2016angeli-qa}).\footnote{
  The code for this system can be found online at \url{https://github.com/gangeli/NaturalLI}
  }
The remainder of this disseration will review natural logic as a formalism, and then describe in depth
  the components of this reasoning system.
